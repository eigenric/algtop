\documentclass[
  a4paper,
  spanish,
  12pt,
]{scrartcl}

\linespread{1.05}
%\setlength{\parindent}{18pt}


%-------------------------------------------------------------------------------
%	PAQUETES
%-------------------------------------------------------------------------------

% Idioma

\usepackage{indentfirst}
% Matemáticas

\usepackage{mathrsfs}

\usepackage{config}
% \usepackage{amsmath, amsthm, amssymb}
% \usepackage{mathtools}
% \usepackage{commath}
% \usepackage{xfrac}


% Fuentes personalizadas para utilizar con XeLaTeX o LuaLaTeX

\usepackage[no-math]{fontspec}
\setmainfont[WordSpace=1.3, RawFeature={+ss06}]{EBGaramond}
\setsansfont[Scale=0.9]{Alegreya Sans}
\setmonofont[Scale=0.75]{Bitstream Vera Sans Mono}

\usepackage[math-style=TeX]{unicode-math}
\setmathfont{Garamond Math}[StylisticSet={3}]


% Configuración de microtype

\defaultfontfeatures{Ligatures=TeX,Numbers=Lining}
\usepackage[activate={true,nocompatibility},final,tracking=true,factor=1100,stretch=10,shrink=10]{microtype}
\SetTracking{encoding={*}, shape=sc}{0}

% Enlaces y colores

\usepackage{hyperref}
\usepackage{xcolor}
\hypersetup{
  colorlinks=true,
  citecolor=,
  linkcolor=,
  urlcolor=blue,
}

% Otros elementos de página

\usepackage{enumitem}
\setlist[enumerate]{leftmargin=-\itemindent, itemsep=0pt}
\setlist[itemize]{leftmargin=-\itemindent, itemsep=0pt}
%\setlist[itemize]{leftmargin=*}
%\setlist[enumerate]{leftmargin=*}

\usepackage[labelfont={sc, sf}, textfont=sf]{caption}

\usepackage{booktabs}
\renewcommand\arraystretch{1.5}

% Tikz

\usepackage{tikz}
\usetikzlibrary{babel}
\usepackage{float}

% Código

\usepackage{listings}
\lstset{
	basicstyle=\ttfamily,%
	breaklines=true,%
	captionpos=b,                    % sets the caption-position to bottom
  tabsize=2,	                   % sets default tabsize to 2 spaces
  frame=lines,
  numbers=left,
  stepnumber=1,
  aboveskip=12pt,
  showstringspaces=false,
  keywordstyle=\bfseries,
  commentstyle=\itshape,
  columns=flexible,
}
\renewcommand{\lstlistingname}{Listado}

% ENTORNOS

\usepackage[theorems, skins, breakable]{tcolorbox}

\tcolorboxenvironment{nth}{
	blanker,
	breakable,
	left=12pt,
	before skip=12pt,
	after skip=12pt,
	borderline west={2pt}{0pt}{500},
	before upper={\parindent 12pt},
}

\tcolorboxenvironment{nprop}{
	blanker,
	breakable,
	left=12pt,
	before skip=12pt,
	after skip=12pt,
	borderline west={2pt}{0pt}{42},
	before upper={\parindent 12pt},
}

\tcolorboxenvironment{ncor}{
	blanker,
	breakable,
	left=12pt,
	before skip=12pt,
	after skip=12pt,
	borderline west={2pt}{0pt}{300},
	before upper={\parindent 12pt},
}

\tcolorboxenvironment{ndef}{
	skin=enhancedmiddle jigsaw,
	frame hidden,
	colback= 36,
	breakable = true,
	break at = -6pt,
	top = 4pt,       % Estos márgenes están un poco a ojo
	bottom = 4pt,
	left= 8pt,
	right = 8pt,
	before skip=8pt, % Normalmente dejamos 12pt, pero
	after skip=8pt,  % aquí tenemos espacio adicional por el fondo
	no borderline,
	borderline west={2pt}{0pt}{42},
	before upper={\parindent 12pt},
}

\tcolorboxenvironment{ejer}{
	skin=enhancedmiddle jigsaw,
	frame hidden,
	colback=50,
	breakable = true,
	break at = -6pt,
	top = 4pt,       % Estos márgenes están un poco a ojo
	bottom = 4pt,
	left= 8pt,
	right = 8pt,
	before skip=8pt, % Normalmente dejamos 12pt, pero
	after skip=8pt,  % aquí tenemos espacio adicional por el fondo
	no borderline,
	borderline west={2pt}{0pt}{500},
	borderline east={2pt}{0pt}{50},
	before upper={\parindent 12pt},
}


% Márgenes
\usepackage[bottom=3.125cm, top=2.5cm, left=3.5cm, right=3.5cm, marginparwidth=70pt]{geometry}

%dealing with (sub/subsub)sections
%\let\raggedsection\centering%Center all sectioningheads
%all levels have something in common, let's save typing:
\RedeclareSectionCommands[beforeskip=-3ex,
afterskip=2ex]{section,subsection,subsubsection}
\addtokomafont{section}{\normalfont\large\textsc}
\RedeclareSectionCommand[beforeskip=-6ex]{section}
\addtokomafont{subsection}{\normalfont\itshape}

%-------------------------------------------------------------------------------
%	CONTENIDO
%-------------------------------------------------------------------------------

\begin{document}

\begin{flushright}
  Ricardo Ruiz Fernández de Alba\vspace{.5em}

  \textit{Probabilidad}

  Doble Grado en Ingeniería Informática y Matemáticas

  \textsc{Universidad de Granada}\vspace{.5em}

  \today\vspace{.5em}
\end{flushright}

\begin{flushleft}
  \scshape\Large Relación de Ejercicios del Tema 3
\end{flushleft}

\begin{ejer}
(1) Sea $X=\mathbb{S}^{2} \cup\left\{x_{0}\right\}$, donde $x_{0} \in \mathbb{R}^{3} \backslash \mathbb{S}^{2}$. En $X$ se considera la topología tal que los entornos de los puntos de $\mathbb{S}^{2}$ son los usuales, y los de $x_{0}$ son de la forma $(V \backslash\{N\}) \cup\left\{x_{0}\right\}$, donde $N=(0,0,1)$ y $V$ es un entorno de $N$ en $\mathbb{S}^{2}$. Demostrar que $X$ es localmente euclídeo, es IIAN, pero no es $T 2$.\\
\end{ejer}

\begin{ejer}
(2) Consideremos el espacio producto $X=\mathbb{R}^{2} \times \mathbb{R}$, donde en $\mathbb{R}^{2}$ se considera la topología usual y en $\mathbb{R}$ la topología discreta. Demostrar que $X$ es localmente euclídeo, es $T 2$, pero no es IIAN.\\
\end{ejer}

\begin{ejer}
(3) Demostrar que los siguientes espacios topológicos no son superficies:\\
(a) $S=\left\{(x, y, z) \in \mathbb{R}^{3}: x^{2}+y^{2}-z^{2}=0\right\}$.\\
(b) $\mathbb{R}^{n}$ con $n \neq 2$.\\
(c) $S=\left\{(x, y) \in \mathbb{R}^{2}: y \geq 0\right\}$.\\
¿Es la unión o intersección de dos superficies en $\mathbb{R}^{3}$ una superficie?\\
\end{ejer}

\begin{ejer}
(4) Una superficie topológica con borde es un espacio topológico $T 2$ en el que todo punto tiene un entorno abierto homeomorfo a un abierto de $\overline{\mathbb{R}_{+}^{2}}=\left\{(x, y) \in \mathbb{R}^{2} \mid y \geq 0\right\}$. Prueba que el cilindro y la banda de Möbius son superficies topológicas con borde.\\
\end{ejer}

\begin{ejer}
(5) Sea $(\widetilde{S}, \pi)$ un recubridor de una superficie topológica $S$. Si $\widetilde{S}$ es IIAN, demostrar que $\widetilde{S}$ es una superficie topológica.\\
\end{ejer}

\begin{ejer}
(6) Sean $S$ una superficie y $f: S \rightarrow \mathbb{R}$ una función continua. Definimos el grafo de $f$ como el espacio topológico

$$
G(f)=\{(x, t) \in S \times \mathbb{R}: t=f(x)\}
$$

con la topología inducida por la topología producto en $S \times \mathbb{R}$. Probar que $G(f)$ es una superficie, que además es compacta si y sólo si lo es $S$.\\
\end{ejer}

\begin{sol}
	Como $S$ es superficie, se tendrá que es un espacio topológico localmente euclídeo, T2 y 2AN.
	Para ver que $G(f)$ es superficie debemos probar estas tres:
	\begin{itemize}
		\item{$G(f)$ es T2}
		
		$G(f) \subset S \times \mathbb{R}$. Como $S$ y $\mathbb{R}$ son T2, se tendrá que $S \times \mathbb{R}$ es T2 con la topología producto y por tanto $G(f)$ herederará T2 con la topología inducida.
		
		\item{$G(f)$ es 2AN}
		
		$G(f) \subset S \times \mathbb{R}$. Como $S$ y $\mathbb{R}$ son 2AN, se tendrá que $S \times \mathbb{R}$ es 2AN con topología producto y por tanto $G(f)$ herederará la propiedad 2AN con la topología inducida.
		
		\item{$G(f)$ es localmente euclídeo}:
		
		Sea $(x,t) \in G(f)$. Entonces $x \in S$ y $t = f(x) \in \mathbb{R}$. Como $S$ es localmente euclídeo, existirá $U \in \mathcal{U}^x$ entorno abierto de $x$ que será homeomorfo (por $\varphi$) a un abierto $V_1\subset \mathbb{R}^2$.
		
		Por otro lado, si consideramos $V_2$ entorno abierto de $f(x)$ tal que $V_2=f(V_1)$, se tendrá que $V_1 \times V_2$ será entorno abierto de $(x,f(x)) \in S \times \mathbb{R}$
		
		De hecho, si consideramos $F: V_1 \times f(V_1) \rightarrow F(V_1 \times f(V_1))$
		dada por $F(x,f(x)) = \varphi(x)$ entonces $F(V_1 \times f(V_1))$ es imagen de un abierto de $S \times \mathbb{R}$ por un homeomorfismo ($\varphi$) y por tanto un abierto.
		
		
	\end{itemize}
	
\end{sol}

\begin{ejer}
(7) Encontrar un atlas de $\mathbb{R}^{2}$, esto es, una familia de cartas en $\mathbb{R}^{2} \mathbb{P}^{2}$ cuyos entornos coordenados recubran $\mathbb{R} \mathbb{P}^{2}$.\\
\end{ejer}

\begin{ejer}
(8) Prueba que la característica de Euler de la suma conexa de dos superficies compactas es igual a la suma de sus características de Euler menos dos.\\
\end{ejer}

\begin{ejer}
(9) Calcula la característica de Euler de la suma conexa de un plano proyectivo y $n$ toros.\\
\end{ejer}

\begin{sol}
$$
\chi(\mathbb{R}P^2_n \# \mathbb{T}_n) = \chi(\mathbb{R}P^2_n) + \chi(T)_n - 2 = 2 - 1 + 2 - 2n - 2 = 1 -2n
$$
\end{sol}

\begin{ejer}
(10) Estudiar la orientabilidad de $S_{1} \# S_{2}$ a partir de la de $S_{1}$ y de $S_{2}$.\\
\end{ejer}

\begin{sol}
La idea de este ejercicio se basa en que toda superficie es homeomorfa a una de las 3 superficies modelo.
Tenemos por tanto 6 casos:

$\chi(\mathbb{S}^2 \# \mathbb{S}^2)$, $\chi(\mathbb{S}^2 \# \mathbb{T}_n)$, $\chi(\mathbb{T}_n \# \mathbb{T}_n)$

Como la suma conexa de orientables, es orientable, esta superficie es orientable.

$\chi(\mathbb{S}^2 \# \mathbb{R}P^2_n)$, $\chi(\mathbb{T}_n \# \mathbb{R}P^2_n)$, $\chi(\mathbb{R}P^2_n \# \mathbb{R}P^2_n)$

Esta no es orientable porque el plano proyectivo no lo es.

\end{sol}

\begin{ejer}
(11) Para cada una de las siguientes presentaciones de superficies, calcula la característica de Euler y determina a cual de las superficies modelo es homeomorfa:\\
a) $\left\langle a, b, c ; a b a c b^{-1} c^{-1}\right\rangle$.\\
b) $\left\langle a, b, c ; a b c a^{-1} b^{-1} c^{-1}\right\rangle$.\\
c) $\left\langle a, b, c, d ; a b c d c a^{-1} b d^{-1}\right\rangle$.\\
d) $\left\langle a, b, c, d, e ; a b a^{-1} c d b^{-1} c^{-1} e d^{-1} e^{-1}\right\rangle$.\\
e) $\left\langle a, b, c, d, e, f ; a b c a d b^{-1} e f c e^{-1} d f^{-1}\right\rangle$.\\
f) $\left\langle a, b, c, d, e, f ; a b c, b d e, c^{-1} d f, e^{-1} f a\right\rangle$.\\
g) $\langle a, b, c, d, e, f, g, h, i, j, k, l, m, n, o$;\\
$\left.a b c, b d e, d f g, f h i, h a j, c^{-1} k l, e^{-1} m n, g^{-1} o k^{-1}, i^{-1} l^{-1} m^{-1}, j^{-1} n^{-1} o^{-1}\right\rangle$.\\
h) $\left\langle a, b, c, d, e, f ; a b c, b d e, c^{-1} d f, e^{-1} f a\right\rangle$.\\
\end{ejer}

\begin{ejer}
(12) Clasificar la suma conexa de las superficies representadas en los apartados a) y b) del ejercicio anterior.\\
\end{ejer}

\begin{ejer}
(13) Demostrar que toda superficie compacta y conexa es homeomorfa a una y sólo una de las siguientes superficies:

$$
\mathbb{S}^{2}, \quad \mathbb{T}_{n}, \quad \mathbb{R P}^{2}, \quad K, \quad \mathbb{T}_{n} \# \mathbb{R} \mathbb{P}^{2}, \quad \mathbb{T}_{n} \# K
$$
\end{ejer}

\begin{ejer}
(14) Identificar, salvo homeomorfismos, las superficies compactas y conexas con característica de Euler igual a -2.\\
\end{ejer}

\begin{ejer}
(15) Sea $S$ una superficie compacta y conexa. Probar que $\chi(S) \geq-2$ si y sólo si $S$ tiene una presentación poligonal $\mathscr{P}=\langle A ; W\rangle$ donde $A$ tiene exactamente 4 elementos.\\
\end{ejer}

\begin{ejer}
(16) Discutir de forma razonada si cada par de las siguientes superficies compactas y conexas son o no homeomorfas entre sí:\\
(a) $S_{1}$ tiene por presentación poligonal a $\left\langle a, b, c, d ; a b c d a d^{-1} c b^{-1}\right\rangle$.\\
(b) $S_{2}$ cumple $\chi\left(S_{2}\right) \geq 0$ y $\Pi_{1}\left(S_{2}\right)$ no es abeliano.\\
(c) $S_{3}$ cumple que $\Pi_{1}\left(S_{3}\right) \cong F(a, b, c) / N\left\{a c b c b a^{-1}\right\}$.\\
\end{ejer}

\begin{ejer}
(17) Obtener la presentación poligonal canónica de la superficie $S_{1}$ del ejercicio anterior efectuando para ello las transformaciones que sean necesarias.\\
\end{ejer}

\begin{ejer}
(18) Sea $S$ la superficie compacta y conexa que admite una presentación poligonal de la forma

$$
\left\langle a, b, c, d, e ; a b^{-1} c-d a^{-1} e b c^{-1}-\right\rangle
$$

donde cada guión - está ocupado por una única etiqueta. Completar la palabra para que $S$ sea homeomorfa a:\\
(a) $\mathbb{T} \# \mathbb{T}$.\\
(b) $\mathbb{R} \mathbb{P}^{2} \# \mathbb{R} \mathbb{P}^{2} \# \mathbb{R} \mathbb{P}^{2} \# \mathbb{R} \mathbb{P}^{2}$.\\
(c) La superficie modelo con grupo fundamental abelianizado isomorfo a $\mathbb{Z}_{2} \times \mathbb{Z}^{4}$.\\
\end{ejer}

\begin{ejer}
(19) Determina los espacios topológicos compactos $Y$ para los que existe una aplicación recubridora $\pi: Y \rightarrow X$ en cada uno de los siguientes casos:\\
(a) $X=\mathbb{S}^{2}$.\\
(b) $X=\mathbb{R} \mathbb{P}^{2}$.\\
(c) $X=\mathbb{S}^{1} \times \mathbb{S}^{1}$.
\end{ejer}


\end{document}
