\documentclass[
  a4paper,
  spanish,
  12pt,
]{scrartcl}

\linespread{1.05}
%\setlength{\parindent}{18pt}


%-------------------------------------------------------------------------------
%	PAQUETES
%-------------------------------------------------------------------------------

% Idioma

\usepackage{indentfirst}
% Matemáticas

\usepackage{mathrsfs}

\usepackage{config}
% \usepackage{amsmath, amsthm, amssymb}
% \usepackage{mathtools}
% \usepackage{commath}
% \usepackage{xfrac}


% Fuentes personalizadas para utilizar con XeLaTeX o LuaLaTeX

\usepackage[no-math]{fontspec}
\setmainfont[WordSpace=1.3, RawFeature={+ss06}]{EBGaramond}
\setsansfont[Scale=0.9]{Alegreya Sans}
\setmonofont[Scale=0.75]{Bitstream Vera Sans Mono}

\usepackage[math-style=TeX]{unicode-math}
\setmathfont{Garamond Math}[StylisticSet={3}]


% Configuración de microtype

\defaultfontfeatures{Ligatures=TeX,Numbers=Lining}
\usepackage[activate={true,nocompatibility},final,tracking=true,factor=1100,stretch=10,shrink=10]{microtype}
\SetTracking{encoding={*}, shape=sc}{0}

% Enlaces y colores

\usepackage{hyperref}
\usepackage{xcolor}
\hypersetup{
  colorlinks=true,
  citecolor=,
  linkcolor=,
  urlcolor=blue,
}

% Otros elementos de página

\usepackage{enumitem}
\setlist[enumerate]{leftmargin=-\itemindent, itemsep=0pt}
\setlist[itemize]{leftmargin=-\itemindent, itemsep=0pt}
%\setlist[itemize]{leftmargin=*}
%\setlist[enumerate]{leftmargin=*}

\usepackage[labelfont={sc, sf}, textfont=sf]{caption}

\usepackage{booktabs}
\renewcommand\arraystretch{1.5}

% Tikz

\usepackage{tikz}
\usetikzlibrary{babel}
\usepackage{float}

% Código

\usepackage{listings}
\lstset{
	basicstyle=\ttfamily,%
	breaklines=true,%
	captionpos=b,                    % sets the caption-position to bottom
  tabsize=2,	                   % sets default tabsize to 2 spaces
  frame=lines,
  numbers=left,
  stepnumber=1,
  aboveskip=12pt,
  showstringspaces=false,
  keywordstyle=\bfseries,
  commentstyle=\itshape,
  columns=flexible,
}
\renewcommand{\lstlistingname}{Listado}

% ENTORNOS

\usepackage[theorems, skins, breakable]{tcolorbox}

\tcolorboxenvironment{nth}{
	blanker,
	breakable,
	left=12pt,
	before skip=12pt,
	after skip=12pt,
	borderline west={2pt}{0pt}{500},
	before upper={\parindent 12pt},
}

\tcolorboxenvironment{nprop}{
	blanker,
	breakable,
	left=12pt,
	before skip=12pt,
	after skip=12pt,
	borderline west={2pt}{0pt}{42},
	before upper={\parindent 12pt},
}

\tcolorboxenvironment{ncor}{
	blanker,
	breakable,
	left=12pt,
	before skip=12pt,
	after skip=12pt,
	borderline west={2pt}{0pt}{300},
	before upper={\parindent 12pt},
}

\tcolorboxenvironment{ndef}{
	skin=enhancedmiddle jigsaw,
	frame hidden,
	colback= 36,
	breakable = true,
	break at = -6pt,
	top = 4pt,       % Estos márgenes están un poco a ojo
	bottom = 4pt,
	left= 8pt,
	right = 8pt,
	before skip=8pt, % Normalmente dejamos 12pt, pero
	after skip=8pt,  % aquí tenemos espacio adicional por el fondo
	no borderline,
	borderline west={2pt}{0pt}{42},
	before upper={\parindent 12pt},
}

\tcolorboxenvironment{ejer}{
	skin=enhancedmiddle jigsaw,
	frame hidden,
	colback=50,
	breakable = true,
	break at = -6pt,
	top = 4pt,       % Estos márgenes están un poco a ojo
	bottom = 4pt,
	left= 8pt,
	right = 8pt,
	before skip=8pt, % Normalmente dejamos 12pt, pero
	after skip=8pt,  % aquí tenemos espacio adicional por el fondo
	no borderline,
	borderline west={2pt}{0pt}{500},
	borderline east={2pt}{0pt}{50},
	before upper={\parindent 12pt},
}


% Márgenes
\usepackage[bottom=3.125cm, top=2.5cm, left=3.5cm, right=3.5cm, marginparwidth=70pt]{geometry}

%dealing with (sub/subsub)sections
%\let\raggedsection\centering%Center all sectioningheads
%all levels have something in common, let's save typing:
\RedeclareSectionCommands[beforeskip=-3ex,
afterskip=2ex]{section,subsection,subsubsection}
\addtokomafont{section}{\normalfont\large\textsc}
\RedeclareSectionCommand[beforeskip=-6ex]{section}
\addtokomafont{subsection}{\normalfont\itshape}

%-------------------------------------------------------------------------------
%	CONTENIDO
%-------------------------------------------------------------------------------

\begin{document}

\begin{flushright}
  Ricardo Ruiz Fernández de Alba\vspace{.5em}

  \textit{Probabilidad}

  Doble Grado en Ingeniería Informática y Matemáticas

  \textsc{Universidad de Granada}\vspace{.5em}

  \today\vspace{.5em}
\end{flushright}

\begin{flushleft}
  \scshape\Large Relación de Ejercicios del Tema 3
\end{flushleft}


\begin{ejer}
	Sea $S$ una superficie y $f: S \rightarrow R$ una función continua. Definimos el grafo de $f$ 
	$$
	G(f) = \{ (x,t) \in S \times \mathbb{R}: t = f(x) \}
	$$
Sea $(G(f), \tau_{G(f)})$ e.t, probar que es una superficie y que además es compacta si y sólo si $S$ es compacta.
\end{ejer}

\begin{sol}
Como $S$ es superficie, se tendrá que es un espacio topológico localmente euclídeo, T2 y 2AN.
Para ver que $G(f)$ es superficie debemos probar estas tres:
\begin{itemize}
    \item{$G(f)$ es T2}
    
    $G(f) \subset S \times \mathbb{R}$. Como $S$ y $\mathbb{R}$ son T2, se tendrá que $S \times \mathbb{R}$ es T2 con la topología producto y por tanto $G(f)$ herederará T2 con la topología inducida.
    
    \item{$G(f)$ es 2AN}
    
    $G(f) \subset S \times \mathbb{R}$. Como $S$ y $\mathbb{R}$ son 2AN, se tendrá que $S \times \mathbb{R}$ es 2AN con topología producto y por tanto $G(f)$ herederará la propiedad 2AN con la topología inducida.
    
    \item{$G(f)$ es localmente euclídeo}:
    
    Sea $(x,t) \in G(f)$. Entonces $x \in S$ y $t = f(x) \in \mathbb{R}$. Como $S$ es localmente euclídeo, existirá $U \in \mathcal{U}^x$ entorno abierto de $x$ que será homeomorfo (por $\varphi$) a un abierto $V_1\subset \mathbb{R}^2$.
    
    Por otro lado, si consideramos $V_2$ entorno abierto de $f(x)$ tal que $V_2=f(V_1)$, se tendrá que $V_1 \times V_2$ será entorno abierto de $(x,f(x)) \in S \times \mathbb{R}$
    
    De hecho, si consideramos $F: V_1 \times f(V_1) \rightarrow F(V_1 \times f(V_1))$
    dada por $F(x,f(x)) = \varphi(x)$ entonces $F(V_1 \times f(V_1))$ es imagen de un abierto de $S \times \mathbb{R}$ por un homeomorfismo ($\varphi$) y por tanto un abierto.
    
    
\end{itemize}

\end{sol}


\end{document}
